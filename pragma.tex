\chapter{pragma预处理命令}
\cppinline{@\#@pragma} 命令在 \cppinline{C++}
中是一个预处理指令,用于控制编译器的特定实现行为。它允许开发者向编译器发出特定指令,这些指令通常是编译器特有的,不是
\cppinline{C++} 标准的一部分,但主流的编译器都支持。

\section{语法}
\begin{cpplst}
|\#|pragma pragma-params
_Pragma(string-literal)
\end{cpplst}
一般比较常见的是第一种写法。

\section{pragma STDC}
在\cppinline{C++}中,\cppinline{@\#@pragma STDC} 是一个预处理器指令,它允许你在编译时指定遵循特定版本的 \cppinline{C} 标准。这个指令主要用于 \cppinline{C} 语言的标准兼容性控制,并不是 \cppinline{C++} 标准的一部分,但是某些 \cppinline{C++} 编译器支持它以提供对 \cppinline{C} 代码的支持。

\cppinline{@\#@pragma STDC} 指令通常用于控制 \cppinline{C99} 或者其他 \cppinline{C} 标准特性的启用或禁用。例如,它可以用来控制是否使用 \cppinline{C99} 的数学库函数或者类型定义等。

\subsection{常见用法}
\cppinline{@\#@pragma STDC} 可以有多个不同的选项,以下是一些常见的选项:
\begin{description}
	\item[\cppinline{@\#@pragma STDC FENV_ACCESS ON}]启用对浮点环境的访问。
	\item[\cppinline{@\#@pragma STDC FENV_ACCESS OFF}]禁用对浮点环境的访问。
	\item[\cppinline{@\#@pragma STDC CX_LIMITED_RANGE ON}]开启复数有限范围。
	\item[\cppinline{@\#@pragma STDC CX_LIMITED_RANGE OFF}]关闭复数有限范围。
	\item[\cppinline{@\#@pragma STDC FP_CONTRACT ON}]允许浮点表达式的收缩评估(即优化)。
	\item[\cppinline{@\#@pragma STDC FP_CONTRACT OFF}]禁止浮点表达式的收缩评估。
	\item[\cppinline{@\#@pragma STDC MATH_ERRNO ON}]允许错误信被写入 \cppinline{errno}。
	\item[\cppinline{@\#@pragma STDC MATH_ERRNO OFF}]禁止错误信息被写入 \cppinline{errno}。
	\item[\cppinline{@\#@pragma STDC MATH_INexact ON}]允许非精确警告。
	\item[\cppinline{@\#@pragma STDC MATH_INexact OFF}]禁止非精确警告。
	\item[\cppinline{@\#@pragma STDC NO_LONG_DOUBLE_MATH_FUNCTIONS}]禁用长双精度数学函数。
	\item[\cppinline{@\#@pragma STDC NO_LONG_DOUBLE}]禁用 \cppinline{long double} 类型。
	\item[\cppinline{@\#@pragma STDC NO_THREADS}]禁用线程支持。
\end{description}

\section{pragma once}
\cppinline{@\#@pragma once} 是一个非标准的预处理指令,用于确保头文件在编译过程中只被包含一次。它的作用类似于传统的头文件保护宏,但具有一些优点和不同之处。

头文件通常包含许多其他头文件,可能导致重复包含。为了避免这种重复,传统的做法是使用预处理宏:
\begin{cpplst}
// MyHeader.h
|\#|ifndef MYHEADER_H
|\#|define MYHEADER_H

// 头文件内容

|\#|endif // MYHEADER_H
\end{cpplst}

\cppinline{@\#@pragma once} 是一种更简洁的方法,只需在头文件的开始处添加一个 \cppinline{@\#@pragma once} 指令即可:
\begin{cpplst}
// MyHeader.h
|\#|pragma once

// 头文件内容

\end{cpplst}

\section{pragma pack}
\cppinline{@\#@pragma pack}
预处理指令用于控制结构体、联合体和类的内存对齐(alignment)。在\cppinline{C}和\cppinline{C++}中,编译器通常会根据目标平台的要求对数据进行对齐,以提高内存访问的效率。然而,有时为了特定的需求,例如减少内存占用或满足硬件接口要求,需要手动控制这些对齐方式。这时就可以使用
\cppinline{@\#@pragma pack}。
\subsection{语法}
\begin{enumerate}
	\item \cppinline{@\#@pragma pack(arg)}
	\item \cppinline{@\#@pragma pack()}
	\item \cppinline{@\#@pragma pack(push)}
	\item \cppinline{@\#@pragma pack(push, arg)}
	\item \cppinline{@\#@pragma pack(pop)}
\end{enumerate}
\vfill

\subsection{示例}
\begin{cpplst}
|\#|include <iostream>

// 默认对齐方式
struct MyStruct {
    char a;    // 1 字节
    int b;     // 4 字节
    double c;  // 8 字节
};

// 改变对齐方式为1字节
#pragma pack(push, 1)
struct MyPackedStruct {
    char a;    // 1 字节
    int b;     // 4 字节
    double c;  // 8 字节
};
#pragma pack(pop)

int main() {
    std::cout << "Size of MyStruct: "
              << sizeof(MyStruct) << std::endl;
    std::cout << "Size of MyPackedStruct: "
              << sizeof(MyPackedStruct) << std::endl;

    return 0;
}
\end{cpplst}
