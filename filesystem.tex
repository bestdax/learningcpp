\chapter{文件系统}
\cppinline{filesystem}库是在\cppinline{C++17}中引入的,这是一个用于操作文件系统及其组件的库,包括路径、常规文件和目录等。文件系统库最初是作为
\cppinline{Boost} 库的一部分开发的,后来成为 \cppinline{ISO/IEC TS 18822:2015} 技术规范,并最终在 \cppinline{C++17} 中合并为标准部分。

\section{功能概述}
文件系统库提供了一系列的功能,包括但不限于:
\begin{itemize}[label=\daxfile]
	\item 路径操作
	\item 文件和目录的创建、复制、移动和删除
	\item 检查文件存在性、类型和权限
	\item 获取和设置文件的最后写入时间
	\item 符号链接的创建和读取
\end{itemize}

\section{核心概念}
\subsection{文件类型}
\begin{description}
	\item[\faFile 文件\cppinline{file}] 文件是计算机存储的一种基本单元,它包含数据或信息。文件可以是文本文档、图片、程序、音频、视频等。在操作系统中,文件通常有一个名称和路径,并且具有特定的属性,如大小、权限、创建时间等。
	\item[\daxfolder 目录\cppinline{directory}] 文件系统中的一种特殊文件,用于组织和管理文件。目录可以包含文件和其他子目录,形成层次结构(树结构)。目录帮助用户和系统更有效地管理和访问文件。根目录是文件系统的顶层目录,所有文件和目录都可以追溯到根目录。
	\item[\daxfile 普通文件\cppinline{regular file}] 常规文件是文件系统中最常见的文件类型,它包含用户数据或程序数据。常规文件可以是文本文件、二进制文件、图像文件、音频文件等。与其他特殊文件(如目录、设备文件、符号链接等)不同,常规文件直接包含数据。
	\item [\faLink 符号链接\cppinline{symbolic link}]符号链接(符号链接或软链接)是文件系统中的一种特殊文件,它包含指向另一个文件或目录的路径。符号链接本身不包含数据,而是一个引用指向目标文件或目录。符号链接允许用户创建快捷方式,方便地访问或管理文件和目录。符号链接与目标文件或目录之间是独立的,如果目标被删除或移动,符号链接会变成断开的链接。
	\item [\faFile* 特殊文件\cppinline{special file}]block, character, fifo, socket等。
\end{description}

\subsection{路径}
\begin{description}
	\item[\faRoad 绝对路径\cppinline{absolute path}]明确标识文件位置的路径。
	\item[\faRoad 规范路径\cppinline{canonical path}]不包含符号链接、\cppinline{.} 或 \cppinline{..} 元素的绝对路径。
	\item[\faRoad 相对路径\cppinline{relative
	path}]相对于文件系统中某个位置标识文件位置的路径。当前目录是\cppinline{.},上层目录为\cppinline{..}。
\end{description}

\section{路径}
\begin{longtable}{LL}\toprule
	\bf{\hfill 成员函数\hfill} & \bf{\hfill 功能\hfill}             \\\midrule
	\endfirsthead
	\multicolumn{2}{l}{接上页}                                      \\
	\toprule
	\bf{\hfill 成员函数\hfill} & \bf{\hfill 功能\hfill}             \\\midrule
	\endhead
	\bottomrule
	\multicolumn{2}{r}{接下页}
	\endfoot
	\endlastfoot
	preferred\_separator       & 当前系统的路径分隔符               \\
	operator\textbackslash=    & 用\textbackslash 合并两个路径      \\
	operator+=                 & 合并两个路径(不添加\textbackslash) \\
	clear                      & 清除路径对象的内容                 \\
	make\_preferred            & 用指定的分隔符来改变路径           \\
	remove\_filename           & 删除路径中的文件名称               \\
	replace\_filename          & 替换路径中的文件名称               \\
	replace\_extension         & 替换路径中的文件后缀名             \\
	parent\_path               & 返回上一级目录                     \\
	filename                   & 返回文件名                         \\
	stem                       & 返回文件名(不包含后缀)             \\
	extension                  & 返回文件后缀名                     \\
	empty                      & 检查路径是否为空                   \\
	has\_relative\_path        & 检查是否有相对路径                 \\
	has\_parent\_path          & 检查是否有上层目录                 \\
	has\_filename              & 检查是否有文件名                   \\
	has\_stem                  & 检查是否有文件名(不含后缀)         \\
	has\_extension             & 检查是否有后缀名                   \\
	is\_absolute               & 检查是否为绝对路径                 \\
	is\_relative               & 检查是否为相对路径                 \\
	swap                       & 交换两个路径                       \\
	begin / end                & 迭代器                             \\
	\bottomrule
\end{longtable}
