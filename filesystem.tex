\chapter{文件系统}
\cppinline{filesystem}库是在\cppinline{C++17}中引入的,这是一个用于操作文件系统及其组件的库,包括路径、常规文件和目录等。文件系统库最初是作为
\cppinline{Boost} 库的一部分开发的,后来成为 \cppinline{ISO/IEC TS 18822:2015} 技术规范,并最终在 \cppinline{C++17} 中合并为标准部分。

\section{功能概述}
文件系统库提供了一系列的功能,包括但不限于:
\begin{itemize}[label=\daxfile]
	\item 路径操作
	\item 文件和目录的创建、复制、移动和删除
	\item 检查文件存在性、类型和权限
	\item 获取和设置文件的最后写入时间
	\item 符号链接的创建和读取
\end{itemize}

\section{核心概念}
\subsection{文件类型}
\begin{description}
	\item[\faFile 文件\cppinline{file}] 文件是计算机存储的一种基本单元,它包含数据或信息。文件可以是文本文档、图片、程序、音频、视频等。在操作系统中,文件通常有一个名称和路径,并且具有特定的属性,如大小、权限、创建时间等。
	\item[\daxfolder 目录\cppinline{directory}] 文件系统中的一种特殊文件,用于组织和管理文件。目录可以包含文件和其他子目录,形成层次结构(树结构)。目录帮助用户和系统更有效地管理和访问文件。根目录是文件系统的顶层目录,所有文件和目录都可以追溯到根目录。
	\item[\daxfile 普通文件\cppinline{regular file}] 常规文件是文件系统中最常见的文件类型,它包含用户数据或程序数据。常规文件可以是文本文件、二进制文件、图像文件、音频文件等。与其他特殊文件(如目录、设备文件、符号链接等)不同,常规文件直接包含数据。
	\item [\faLink 符号链接\cppinline{symbolic link}]符号链接(符号链接或软链接)是文件系统中的一种特殊文件,它包含指向另一个文件或目录的路径。符号链接本身不包含数据,而是一个引用指向目标文件或目录。符号链接允许用户创建快捷方式,方便地访问或管理文件和目录。符号链接与目标文件或目录之间是独立的,如果目标被删除或移动,符号链接会变成断开的链接。
	\item [\faFile* 特殊文件\cppinline{special file}]block, character, fifo, socket等。
\end{description}

\subsection{路径}
\begin{description}
	\item[\faRoad 绝对路径\cppinline{absolute path}]明确标识文件位置的路径。
	\item[\faRoad 规范路径\cppinline{canonical path}]不包含符号链接、\cppinline{.} 或 \cppinline{..} 元素的绝对路径。
	\item[\faRoad 相对路径\cppinline{relative
	path}]相对于文件系统中某个位置标识文件位置的路径。当前目录是\cppinline{.},上层目录为\cppinline{..}。
\end{description}

\section{路径}
\begin{longtable}{LL}\toprule
	\bf{\hfill 成员函数\hfill} & \bf{\hfill 功能\hfill}                              \\\midrule
	\endfirsthead
	\multicolumn{2}{l}{接上页}                                                       \\
	\toprule
	\bf{\hfill 成员函数\hfill} & \bf{\hfill 功能\hfill}                              \\\midrule
	\endhead
	\bottomrule
	\multicolumn{2}{r}{接下页}
	\endfoot
	\bottomrule
	\endlastfoot
	operator=                  & 赋值另一个路径对象                                  \\
	assign                     & 赋值路径内容                                        \\
	append                     & 用目录分隔符追加元素到路径                          \\
	operator/=                 & 用目录分隔符追加元素到路径                          \\
	concat                     & 不引入目录分隔符连接两个路径                        \\
	operator+=                 & 连接两个路径,不引入目录分隔符                      \\
	clear                      & 清除路径内容                                        \\
	make\_preferred            & 转换目录分隔符为首选目录分隔符                      \\
	remove\_filename           & 移除路径中的文件名组件                              \\
	replace\_filename          & 替换路径中的最后一个组件为另一个路径                \\
	replace\_extension         & 替换路径中的扩展名                                  \\
	swap                       & 交换两个路径对象                                    \\
	c\_str                     & 返回路径的原生字符串表示形式                        \\
	native                     & 返回路径的原生格式字符串                            \\
	operator string\_type      & 返回字符串原生格式                                  \\
	string                     & 返回字符串原生格式,类型为\texttt{std::string}      \\
	wstring                    & 返回字符串原生格式,类型为\texttt{std::wstring}     \\
	u8string                   & 返回字符串UTF-8编码,类型为\texttt{std::string}     \\
	u16string                  & 返回字符串UTF-16编码,类型为\texttt{std::u16string} \\
	u32string                  & 返回字符串UTF-32编码,类型为\texttt{std::u32string} \\
	generic\_string            & 返回字符串通用格式                                  \\
	generic\_wstring           & 返回宽字符串,使用通用格式                          \\
	generic\_u8string          & 返回UTF-8字符串形式,使用通用格式                   \\
	generic\_u16string         & 返回UTF-16字符串形式,使用通用格式                  \\
	generic\_u32string         & 返回UTF-32字符串形式,使用通用格式                  \\
	compare                    & 对两个路径的词法表示进行字典序比较                  \\
	lexically\_normal          & 将路径转换为规范化形式                              \\
	lexically\_relative        & 将路径转换为相对形式                                \\
	lexically\_proximate       & 将路径转换为近似形式                                \\
	root\_name                 & 返回路径的根名称,如果存在                          \\
	root\_directory            & 返回路径的根目录,如果存在                          \\
	root\_path                 & 返回路径的根路径,如果存在                          \\
	relative\_path             & 返回相对于根路径的路径                              \\
	parent\_path               & 返回父路径                                          \\
	filename                   & 返回路径中的文件名组件                              \\
	stem                       & 返回路径中的文件干(不包括最终扩展名)              \\
	extension                  & 返回路径中的文件扩展名组件                          \\
	empty                      & 检查路径是否为空                                    \\
	has\_root\_path            & 检查路径是否有根路径                                \\
	has\_root\_name            & 检查路径是否有根名称                                \\
	has\_root\_directory       & 检查路径是否有根目录                                \\
	has\_relative\_path        & 检查路径是否有相对路径                              \\
	has\_parent\_path          & 检查路径是否有父路径                                \\
	has\_filename              & 检查路径是否有文件名                                \\
	has\_stem                  & 检查路径是否有文件干                                \\
	has\_extension             & 检查路径是否有扩展名                                \\
	is\_absolute               & 检查路径是否为绝对路径                              \\
	is\_relative               & 检查路径是否为相对路径                              \\
	begin                      & 返回路径的迭代器,开始遍历                          \\
	end                        & 返回路径迭代器,结束遍历                            \\
\end{longtable}

\section{非成员函数}
\begin{longtable}{LL}
	\bf{\hfill 非成员函数\hfill} & \bf{\hfill 功能\hfill}                 \\\midrule
	\endfirsthead
	\multicolumn{2}{l}{接上页}                                            \\
	\toprule
	\bf{\hfill 非成员函数\hfill} & \bf{\hfill 功能\hfill}                 \\\midrule
	\endhead
	\bottomrule
	\multicolumn{2}{r}{接下页}
	\endfoot
	\bottomrule
	\endlastfoot
	absolute                     & 组合一个绝对路径                       \\
	canonical                    & 组合一个规范路径                       \\
	weakly\_canonical            & 组合一个弱规范路径                     \\
	relative                     & 组合一个相对路径                       \\
	proximate                    & 组合一个近似路径                       \\
	copy                         & 复制文件或目录                         \\
	copy\_file                   & 复制文件内容                           \\
	copy\_symlink                & 复制一个符号链接                       \\
	create\_directory            & 创建新目录                             \\
	create\_directories          & 创建新目录(包括所有必要的父目录)     \\
	create\_hard\_link           & 创建一个硬链接                         \\
	create\_symlink              & 创建一个符号链接                       \\
	create\_directory\_symlink   & 创建一个目录符号链接                   \\
	current\_path                & 返回或设置当前工作目录                 \\
	exists                       & 检查路径是否指向现有的文件系统对象     \\
	equivalent                   & 检查两个路径是否指向同一个文件系统对象 \\
	file\_size                   & 返回文件的大小                         \\
	hard\_link\_count            & 返回指向特定文件的硬链接数量           \\
	last\_write\_time            & 获取或设置数据最修改的时间             \\
	permissions                  & 修改文件访问权限                       \\
	read\_symlink                & 获取符号链接的目标                     \\
	remove                       & 移除文件或空目录                       \\
	remove\_all                  & 递归移除文件或目录及其所有内容         \\
	rename                       & 移动或重命名文件或目录                 \\
	resize\_file                 & 通过截断或零填充改变普通文件的大小     \\
	space                        & 确定文件系统上的可用空闲空间           \\
	status                       & 确定文件属性                           \\
	symlink\_status              & 确定文件属性,检查符号链接的目标       \\
	temp\_directory\_path        & 返回适合临时文件的目录                 \\
	is\_block\_file              & 检查给定路径是否指向块设备             \\
	is\_character\_file          & 检查给定路径是否指向字符设备           \\
	is\_directory                & 检查给定路径是否指向目录               \\
	is\_empty                    & 检查给定路径是否指向空文件或目录       \\
	is\_fifo                     & 检查给定路径是否指向命名管道           \\
	is\_other                    & 检查参数是否指向其他文件               \\
	is\_regular\_file            & 检查参数是否指向普通文件               \\
	is\_socket                   & 检查参数是否指向命名IPC套接字          \\
	is\_symlink                  & 检查参数是否指向符号链接               \\
	status\_known                & 检查文件状态是否已知                   \\
\end{longtable}
