\chapter{Lambda函数}

\cppinline{C++}中的\cppinline{Lambda}函数是匿名函数(即没有名字的函数),它可以在代码中定义和使用,特别适合用于需要短期使用的函数或者传递函数作为参数的场景。Lambda函数在\cppinline{C++11}中引入,极大地增强了\cppinline{C++}的函数式编程能力。

\section{基本语法}
\cppinline{Lambda}函数的基本语法如下:
\begin{cpplst}
|[|捕获列表|]|(参数列表) -> 返回类型 {
    函数体
}
\end{cpplst}

\subsection{捕获列表 \texttt{[ ]}}
捕获列表用于指定\cppinline{Lambda}函数中可以使用的外部变量。可以按值捕获(即拷贝外部变量),或按引用捕获(即使用外部变量的引用)。捕获列表的具体形式包括:
\begin{itemize}
	\item \cppinline{[ ]}:不捕获任何外部变量。
	\item \cppinline{[=]}:按值捕获所有外部变量。
	\item \cppinline{[&]}:按引用捕获所有外部变量。
	\item \cppinline{[this]}:捕获当前类的\cppinline{this}指针(通常在成员函数中使用)。
	\item \cppinline{[x, &y]}:按值捕获变量\cppinline{x},按引用捕获变量\cppinline{y}。
\end{itemize}

\subsection{参数列表 \texttt{( )}}
参数列表与普通函数类似,用于指定传递给\cppinline{Lambda}函数的参数。如果没有参数,可以省略。

\subsection{返回类型 \texttt{->}}
返回类型可以省略,编译器会自动推导。如果\cppinline{Lambda}函数体中只有一个返回语句,则可以不写返回类型,编译器会自动推导。如果有多个返回语句且类型不一致,必须显式指定返回类型。

\subsection{函数体 \texttt{\{ \}}}
函数体是\cppinline{Lambda}函数的主体,包含具体的代码逻辑。

\section{示例}
以下是一些使用\cppinline{Lambda}函数的简单示例:

\subsection[简单的Lambda函数]{简单的\cppinline{Lambda}函数}
\begin{cpplst}
auto sum = [](int a, int b) -> int {
    return a + b;
};
int result = sum(3, 4);  // result = 7
\end{cpplst}

\subsection{捕获部变量}
\begin{cpplst}
int x = 10;
auto multiply = [x](int y) {
    return x * y;
};
int result = multiply(5);  // result = 50
\end{cpplst}

\subsection{按引用捕获外部变量}
\begin{cpplst}
int x = 10;
auto add = [&x](int y) {
    x += y;
};
add(5);  // x = 15
\end{cpplst}

\subsection{使用Lambda函数排序}
\begin{cpplst}
std::vector<int> vec = {1, 3, 2, 5, 4};
std::sort(vec.begin(), vec.end(), [](int a, int b) {
    return a > b;  // 降序排序
});
\end{cpplst}

\section{总结}
\cppinline{Lambda}函数为\cppinline{C++}提供了更加灵活和简洁的函数定义方式,尤其在需要传递小型函数作为参数的场景下显得非常方便。它结合了捕获外部变量的能力,可以有效减少代码冗余并提高代码的可读性。
