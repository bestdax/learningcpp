\chapter{C++中的<=>运算符}

C++20 引入了 \cppinline{<=>} 运算符,也称为“三路比较运算符”或“太空船运算符”(spaceship operator)。这个运算符主要用于简化和统一比较操作,为对象之间的排序和比较提供了一致的接口。

\section{基本概念}
\cppinline{<=>} 运算符用于同时处理小于(\cppinline{<})、等于(\cppinline{==})和大于(\cppinline{>})的比较。它返回一个类型的结果,该类型表示两个对象之间的相对顺序:

\begin{itemize}
	\item 如果左侧操作数小于右侧操作数,返回一个负值\\(通常是\cppinline{std::strong_ordering::less})。
	\item 如果左侧操作数等于右侧操作数,返回零\\(通常是\cppinline{std::strong_ordering::equal})。
	\item 如果左侧操作数大于右侧操作数,返回一个正值\\(通常是\cppinline{std::strong_ordering::greater})。
\end{itemize}

\section{运算符返回类型}
根据比较的类型不同,\cppinline{<=>} 运算符有种可能的返回类型:

\begin{itemize}
	\item \cppinline{std::strong_ordering}:用于强排序的比较,可以和所有其他类型进行隐式转换。
	\item \cppinline{std::weak_ordering}:用于弱排序的比较,支持等价性但不要求全序。
	\item \cppinline{std::partial_ordering}:用于部分排序的比较,允许有些元素不可比。
	\item \cppinline{std::strong_equality} 和 \cppinline{std::weak_equality}:用于仅进行等价性比较的场景。
\end{itemize}

\section{示例}
以下是一个简单的示例,展示如何在类中使用 \cppinline{<=>} 运算符:

\begin{cpplst}
|\#|include <iostream>
|\#|include <compare>

struct Point {
    int x, y;

    // 使用 <=> 运算符自动生成其他比较运算符
    auto operator<=>(const Point&) const = default;
};

int main() {
    Point p1{1, 2};
    Point p2{1, 3};

    if (p1 < p2)
        std::cout << "p1 < p2\n";
    else if (p1 == p2)
        std::cout << "p1 == p2\n";
    else
        std::cout << "p1 > p2\n";
}
\end{cpplst}

在这个例子中,\cppinline{operator<=>} 被定义为 \cppinline{default},它会自动生成 \cppinline{operator<}、\cppinline{operator<=}、\cppinline{operator>}、\cppinline{operator>=} 和 \cppinline{operator==} 等比较运算符。

\section{自动生成比较运算符}
定义 \cppinline{<=>} 运算符后,编译器会自动生成以下所有比较运算符:

\begin{itemize}
	\item \cppinline{operator<}
	\item \cppinline{operator<=}
	\item \cppinline{operator>}
	\item \cppinline{operator>=}
	\item \cppinline{operator==}
	\item \cppinline{operator!=}
\end{itemize}

这简化了代码,避免手动定义多个比较运算符。

\section{自定义 <=> 运算符}
你也可以自定义 \cppinline{<=>} 运算符以实现复杂的比较逻辑。下面是一个自定义的例子:

\begin{cpplst}
|\#|include <iostream>
|\#|include <compare>

struct Point {
    int x, y;

    auto operator<=>(const Point& other) const {
        if (auto cmp = x <=> other.x; cmp != 0)
            return cmp;
        return y <=> other.y;
    }
};

int main() {
    Point p1{1, 2};
    Point p2{1, 3};

    if (p1 < p2)
        std::cout << "p1 < p2\n";
    else if (p1 == p2)
        std::cout << "p1 == p2\n";
    else
        std::cout << "p1 > p2\n";
}
\end{cpplst}

在这个例子中,\cppinline{x} 和 \cppinline{y} 分别进行比较,只有在 \cppinline{x} 值相等时才会比较 \cppinline{y}。

\section{优点}
\begin{itemize}
	\item \textbf{简化代码}:通过自动生成多个比较运算符,减少了重复代码。
	\item \textbf{一致性}:确保所有的比较运算符具有一致的语义。
	\item \textbf{性能优化}:在某些情况下,编译器可以对比较操作进行优化。
\end{itemize}

\section{使用注意事项}
\begin{itemize}
	\item \textbf{返回类型选择}:应根据需求选择合适的返回类型,使用\\\cppinline{std::strong_ordering} 可确保全序性,而 \cppinline{std::partial_ordering} 可用于部分有序的集合。
	\item \textbf{默认实现}:对于结构体和类中简单的成员变量比较,可以直接使用 \cppinline{default} 实现,但在有复杂逻辑时,可能需要自定义比较逻辑。
\end{itemize}

总的来说,\cppinline{<=>} 运算符是C++20中一个强大的新工具,能够显著简化比较运算符的实现,增强代码的可读性和一致性。
