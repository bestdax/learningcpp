\chapter{哈希常数}

在计算机科学和工程中,常常会用到各种常数来优化算法的性能。本文将介绍哈希常数 \cppinline{0x9e3779b9},它在一些著名的哈希算法中扮演了重要角色。

\section{常数的历史}

\cppinline{0x9e3779b9} 是一个特定的常数,最初由日本计算机科学家 \textit{Peter K. T.} 在其设计的 \textit{Jenkins hash functions} 中使用。这个常数是基于 \textit{Golden Ratio} 的一个变种。其主要目的是在散列计算中提供高效的均匀分布,从而减少冲突的概率。

\section{常数的数学背景}

\cppinline{0x9e3779b9} 以十六进制表示,等于十进制的 \cppinline{2654435769}。该常数与 \textit{Golden Ratio}
相关,因为它是 {$2^{32}$} 除以 \textit{Golden Ratio}(约为 1.618033988749895)的结果。其精确的计算方式为:

\[
\cppinline{0x9e3779b9} = \left\lfloor \frac{2^{32}}{\phi} \right\rfloor
\]

其中 \(\phi\) 为黄金分割数 \(\frac{1 + \sqrt{5}}{2}\)。

\section{在哈希算法中的应用}

\cppinline{0x9e3779b9} 被广泛应用于各种哈希算法和散列函数中。它的使用主要体现在以下几个方面:

\subsection{Jenkins' Hash Functions}

在 Jenkins 的 \textit{One-at-a-Time Hash} 函数中,\cppinline{0x9e3779b9} 被用作初始化常数,通过对数据进行加密处理,确保哈希值的均匀分布。

\subsection{MurmurHash}

\cppinline{0x9e3779b9} 也被用作 MurmurHash 的常数之一,这是一种非加密的散列函数,用于高速数据散列。MurmurHash 使用这个常数来混合数据位,以达到更好的散列质量。

\subsection{CityHash}

在 Google 的 CityHash 中,这个常数用于提高散列的速度和质量,特别是在处理大规模数据时。它通过不同的位移和混合操作,确保了散列结果的均匀性和分布性。

\section{总结}

\cppinline{0x9e3779b9} 是一个重要的哈希常数,因其与黄金分割数的关系以及在多种哈希算法中的应用,成为了哈希函数设计中的一个键组成部分。它的使用帮助提高了算法的性能和可靠性,特别是在处理大量数据时。

