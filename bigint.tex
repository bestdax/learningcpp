\chapter{大整数的实现}
大整数类(BigInt)是一种用于处理超出标准数据类型范围的整数的自定义数据结构。在C++中实现BigInt类,可以有效地处理大数运算,如加法、乘法等。

\section{设计原理}
BigInt类的设计借鉴了Python中的大整数实现,采用数组存储大整数的每一位,其中每个元素为32位无符号整数。有效数据存储在低30位,高2位用于进位信息。

\subsection{内部存储结构}
使用\cppinline{std::vector<uint32_t>}作为内部存储结构,每个\cppinline{uint32_t}存储大整数的一位,以$2^{30}$为基数。

\section{实现细节}
\subsection{构造函数}
BigInt类提供多种构造方式,包括默认构造、从\cppinline{uint64_t}构造、从字符串构造。

\subsection{加法运算}
加法运算符重载实现两个BigInt对象的加法。通过迭代两个数的每一位,考虑进位,计算每一位的和。

\subsection{乘法运算}
乘法运算符重载实现两个BigInt对象乘法。采用类似于小学乘法的竖式计算方法,逐位相乘并考虑进位。

\subsection{输出流重载}
为了能够输出BigInt对象,重载了输出流运算符\cppinline{<<}。需要将内部的二进制表示转换为十进制字符串。

\section{代码实现}
以下是BigInt类的部分实现代码:

\begin{cpplst}[breakable]
class BigInt {
private:
    std::vector<uint32_t> digits;
    static const uint32_t BASE = 1 << 30;  // 2^30
    static const uint32_t MASK = BASE - 1;  // 低30位掩码

    // 移除前导零
    void trim() {
        // ... 实现细节 ...
    }

public:
    // 默认构造函数
    BigInt() : digits(1, 0) {}

    // 从uint64_t类型整数构造
    BigInt(uint64_t num) {
        while (num > 0) {
            digits.push_back(num & MASK);
            num >>= 30;
        }
        trim();
    }

    // 从字符串构造
    BigInt(const std::string& str) {
        // ... 实现细节 ...
    }

    // 加法运算符重载
    BigInt operator+(const BigInt& other) const {
        BigInt result;
        // ... 实现细节 ...
        return result;
    }

    // 乘法运算符重载
    BigInt operator*(const BigInt& other) const {
        BigInt result;
        // ... 实现细节 ...
        return result;
    }

    // 输出流运算符重载
    friend std::ostream& operator<<(std::ostream& os, const BigInt& bigint) {
        if (bigint.digits.empty()) {
            return os << 0;
        }

        std::string result;
        // ... 转换逻辑 ...
        return os << result;
    }
};
\end{cpplst}


\section{输出流重载(operator<<)}

\subsection{代码实现}
\begin{cpplst}[breakable]
friend std::ostream& operator<<(std::ostream& os, const BigInt& bigint) {
    if (bigint.digits.empty()) {
        return os << 0;
    }

    std::string result;
    BigInt base(BASE);
    BigInt current_value = bigint;

    while (!current_value.digits.empty() && current_value.digits.back() > 0) {
        uint32_t remainder = 0;
        for (int i = current_value.digits.size() - 1; i >= 0; --i) {
            uint64_t temp = (static_cast<uint64_t>(remainder) << 30) + current_value.digits[i];
            current_value.digits[i] = temp / 10;
            remainder = temp % 10;
        }
        result.push_back('0' + remainder);
        current_value.trim();
    }

    std::reverse(result.begin(), result.end());
    os << result;
    return os;
}
\end{cpplst}

\subsection{逻辑步骤解析}
\begin{enumerate}
    \item 处理空对象的情况:如果\cppinline{BigInt}对象的\cppinline{digits}数组为空,则直接输0。
    \item 初始化变量:创建一个用于存储最终十进制字符串表示的\cppinline{result}字符串。
    \item 循环处理大整数:使用\cppinline{while}循环,当\cppinline{current_value}不为空且其\cppinline{digits}数组的最后一个元素大于0时,进入循环。
    \item 逐位计算十进制数:在循环内部,初始化\cppinline{remainder}为0,然后从\cppinline{current\_value.digits}的最高有效位到最低有效位进行遍历。
    \item 更新当前块和余数:通过\cppinline{temp / 10}更新当前块的值,通过\cppinline{temp @\%@ 10}得到新的余数\cppinline{remainder}。
    \item 存储当前位并修剪零:将计算得到的余数\cppinline{remainder}转换为字符并添加到\cppinline{result}字符串的末尾。
    \item 反转字符串并输出:由于从最低有效位开始逐位提取,最终的字符串\cppinline{result}是反向的,需要反转。然后输出到流中。
\end{enumerate}

\section{注意事项}
\begin{itemize}
    \item 在实现乘法和除法时,要特别注意进位和借位的处理。
    \item 输出流重载需要正确处理每一位的基数权值。
    \item 对BigInt类的测试应覆盖各种大小的数值,包括边界情况。
\end{itemize}
