\chapter{折迭表达式}
今天\footnote{2024年08月01日}无意中发现了C++的这个在17标准中引用的新的功能,看了一下非常强悍。先来看一个示例:
\begin{cpplst}
template<typename... Args>
bool all(Args... args) { return (... && args); }

bool b = all(true, true, true, false);
// within all(), the unary left fold expands as
//  return ((true && true) && true) && false;
// b is false
\end{cpplst}
上面的代码会实现将\cppinline{Python}中类似的效果,同样也可以用来轻松实现\cppinline{any}等。

\section{语法结构}
\cppinline{cppreference}网站上将折叠表达式的语法分为以下四类:
\begin{enumerate}
	\item \cppinline{( pack op ... )} 单目右折迭
	\item \cppinline{( ... op pack )} 单目左折迭
	\item \cppinline{( pack op ... op init )} 双目右折迭
	\item \cppinline{( init op ... op pack )} 双目左折迭
\end{enumerate}

\begin{description}
	\item[\cppinline{op}] 一共下列32种双元操作符:\footnote{好不容易把这些符号输入进来}

	      \forcsvlist{\cppinline}{+, -, *, /, @\%@, ^, &, |, =, <, >, <<, >>, +=, -=, *=, /=,@\%@=}

	      \forcsvlist{\cppinline}{^=, &=, |=, <<=, >>=, ==, !=, <=, >=, &&, ||, @\texttt{,}@, .*, ->*}

	      注意在双元折叠模式下两个操作符必须是一样的。
	\item[\cppinline{pack}] 一个包含未展开的参数包的表达式,并且在顶层不包含优先级低于强制类型转换的运算符(即\cppinline{cast})。
	\item[\cppinline{init}]
	      一个表达式,它不包含未展开的参数包,并且在顶层不包含优先级低于强制类型转换的运算符(即\cppinline{cast})。
\end{description}
\emph{注意外围的圆括号,这是表达式的一部分。}

\section{说明}
\noindent
\begin{description}
	\item[单目右折迭] \cppinline{E op ...}展开成\cppinline{(E@$_1$@ op (... op (E@$_{N-1}$@ op E@$_N$@)))}
	\item[单目左折迭] \cppinline{... op E}展开成\cppinline{(((E@$_1$@ op E@$_2$@) op ...) op E@$_N$@)}
	\item[双目右折迭] \cppinline{E op ... op I}展开成\cppinline{(E@$_1$@ op (... op (E@$_{N-1}$@ op (E@$_N$@ op I))))}
	\item[双目左折迭] \cppinline{I op ... op E}展开成\cppinline{((((I op E@$_1$@) op E@$_2$@) op ...) op E@$_N$@)}
\end{description}

当\cppinline{pack}为空时,只有\forcsvlist{\cppinline}{&&, ||, @\texttt{,}@}三种运算符可以使用,并有如下规定:
\begin{description}

	\item[逻辑与\cppinline{&&}]当参数包为空时,逻辑与运算的结果是
	      \cppinline{true}。这是因为逻辑与运算通常用于检查多个条件是否全部为
	      \cppinline{true}。如果没有条件要检查,可以认为没有条件被违反,因此默认结果为 \cppinline{true}。

	\item[逻辑或\cppinline{||}]当参数包为空时,逻辑或运算的结果是
	      \cppinline{false}。这是因为逻辑或运算通常用于检查是否至少有一个条件为
	      \cppinline{true}。如果没有条件要检查,意味着没有任何条件满足,因此默认结果为 \cppinline{false}。

	\item[逗号运算符\cppinline{@,@}]:当参数包为空时,逗号运算符的结果为
	      \cppinline{void()}。这是因为逗号运算符用于顺序执行一系列表达式,如果没有表达式要执行,其结果自然应该是
	      \cppinline{void} 类型,表示没有返回值。
\end{description}

\section{示例}
\begin{cpplst}
// 打印多个参数并在中间在“, ”隔开

template<typename... Args>
void print(Args&&... args)
{
	// Print all arguments separated by ", "
	bool first = true;
	((std::cout << (first ? first = false, "" : ", ") << args), ...) << std::endl;
}
\end{cpplst}

\begin{cpplst}
// 基本上都需要用圆括号括起来

template<typename... Args>
int sum(Args&&... args)
{
//  return (args + ... + 1 * 2);  
// Error: operator with precedence below cast
    return (args + ... + (1 * 2)); // OK
}
\end{cpplst}

\begin{cpplst}
std::vector<int> ivec{};
template<typename... Args>
auto push_back(Args&&... args)
{
	(..., (ivec.push_back(args)));
}

// 此示例中把...与操作调换位置也不影响插入的顺序
\end{cpplst}
